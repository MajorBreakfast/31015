\documentclass[10pt,conference,draftcls]{IEEEtran}
\usepackage{fancyhdr}
%\usepackage[top=3cm, bottom=2cm, left=2cm, right=4cm]{geometry}
\usepackage[english]{babel}
\usepackage[utf8]{inputenc}
\usepackage[noadjust]{cite} % Bibtex package
\usepackage[cmex10]{amsmath}
\usepackage{amssymb}
\usepackage{amsfonts}
\usepackage{graphicx}
\usepackage{float}
%\usepackage{caption}
\usepackage[caption=false,labelfont=sf,textfont=sf]{subfig}
%\usepackage{lmodern}
\graphicspath{{./pictures/}}
%\usepackage{textpos}
\usepackage{color}
\usepackage{xcolor}
\usepackage{listingsutf8}
\usepackage{courier}
\usepackage{epstopdf}
\usepackage{fixltx2e}
\usepackage{pdfpages}
\usepackage{lipsum}
\usepackage{tikz}
\usetikzlibrary{external}\tikzexternalize
\usepackage{pgfplots}
\usepackage{svg}

%\epstopdfsetup{update,prepend,verbose,suffix=-generated} % use suffix because you don't want to accidentally overwrite a file that might have been a pdf source. The epstopdf package manual has more on that.

 
 \usepackage{hyperref}

%palatino
% Palatino for rm and math | Helvetica for ss | Courier for tt
%\usepackage{mathpazo} % math & rm
%\usepackage{mathrsfs}
%\linespread{1.05}        % Palatino needs more leading (space between lines)
%\usepackage[scaled]{helvet} % ss
%\usepackage{courier} % tt
%\normalfont
%\usepackage[T1]{fontenc}
\usepackage{bm}
\definecolor{javared}{rgb}{0.6,0,0} % for strings
\definecolor{javagreen}{rgb}{0.25,0.5,0.35} % comments
\definecolor{javapurple}{rgb}{0.5,0,0.35} % keywords
\definecolor{javadocblue}{rgb}{0.25,0.35,0.75} % javadoc
\definecolor{mauve}{rgb}{0.6,0,0}
%\definecolor{box}{rgb}{0.98,0.98,0.98} % box

%TESTTESTTEST

%setup listings
\lstset{language=C,
  extendedchars=true,
  %language=Octave,                % the language of the code
  basicstyle=\ttfamily\footnotesize,% the size of the fonts that are used for the code
  numbers=left,                   % where to put the line-numbers
  numberstyle=\tiny\color{gray},  % the style that is used for the line-numbers
  stepnumber=2,                   % the step between two line-numbers. If it's 1, each line 
                                  % will be numbered
  numbersep=5pt,                  % how far the line-numbers are from the code
  backgroundcolor=\color{white},      % choose the background color. You must add \usepackage{color}
  showspaces=false,               % show spaces adding particular underscores
  showstringspaces=false,         % underline spaces within strings
  showtabs=false,                 % show tabs within strings adding particular underscores
  frame=single,                   % adds a frame around the code
  rulecolor=\color{black},        % if not set, the frame-color may be changed on line-breaks within not-black text (e.g. comments (green here))
  tabsize=4,                      % sets default tabsize to 2 spaces
  captionpos=b,                   % sets the caption-position to bottom
  breaklines=true,                % sets automatic line breaking
  breakatwhitespace=false,        % sets if automatic breaks should only happen at whitespace
  title=\lstname,                   % show the filename of files included with \lstinputlisting;
                                  % also try caption instead of title
  keywordstyle=\color{blue},          % keyword style
  commentstyle=\color{javagreen},       % comment style
  stringstyle=\color{javared},         % string literal style
  escapeinside={\%*}{*)},            % if you want to add LaTeX within your code
  morekeywords={*,...},              % if you want to add more keywords to the set
  deletekeywords={...}              % if you want to delete keywords from the given language
}
\lstset{literate=%
{æ}{{\ae}}1
{å}{{\aa}}1
{ø}{{\o}}1
{Æ}{{\AE}}1
{Å}{{\AA}}1
{Ø}{{\O}}1
{°}{{${}^o$}}1
{I̅}{{$\overline{\mbox{I}}$}}1
{X̅}{{$\overline{\mbox{X}}$}}1
{V̅}{{$\overline{\mbox{V}}$}}1
{L̅}{{$\overline{\mbox{L}}$}}1
{D̅}{{$\overline{\mbox{D}}$}}1
{C̅}{{$\overline{\mbox{C}}$}}1
{M̅}{{$\overline{\mbox{M}}$}}1
{M̅}{{$\overline{\mbox{M}}$}}1
}

 \lstloadlanguages{% Check Dokumentation for further languages ...
         %[Visual]Basic
         %Pascal
         %C
         %C++
         %XML
         %HTML
         %Java
         VHDL
 }
%\DeclareCaptionFont{blue}{\color{blue}} 

%\captionsetup[lstlisting]{singlelinecheck=false, labelfont={blue}, textfont={blue}}
%\usepackage{caption}
\DeclareCaptionFont{white}{\color{white}}
\DeclareCaptionFormat{listing}{\colorbox[cmyk]{0.43, 0.35,
0.35,0.01}{\parbox{0.98\textwidth}{\hspace{15pt}#1#2#3}}}
\captionsetup[lstlisting]{format=listing,labelfont=white,textfont=white, singlelinecheck=false, margin=0pt, font={bf,footnotesize}}
%\lstset{language=Java}

\addto\captionsdanish{%
  \renewcommand{\abstractname}%
    {Abstract}%
}

%\setcounter{chapter}{1}
\setcounter{section}{0}

%\pagestyle{fancy}

%\lhead{31400 Electromagnetics assignment 3}
%\chead{ 22/11-13 }
%\chead[<ch-even>]{<ch-odd>}	
%\rhead{Carsten Nielsen s123161} %, \\ Søren Krogh Andersen s123369}

%\cfoot[\thepage]{\thepage}

%Macros
%By s�ren:
\newcommand\stdfig[4]{ %width,img,cap,label
\begin{figure}[h!]
\centering
\includegraphics[width=#1\textwidth]{#2}
\caption{#3} \label{#4}
\end{figure}
}
\newcommand\diff{\dot}
\newcommand\ddiff{\ddot}
\newcommand\facit[1]{\underline{\underline{#1}}}
\DeclareUnicodeCharacter{00A0}{ }

\begin{document}
\section*{Projektets forudsætning}
Ved projektets start troede vi at vi skulle fokusere på at dæmpe mekanisk støj på accelerometeret, men det viste sig at "støjen" skyldtes kommunikationsfejl.
Dette medførte at vi var nødt til at ændre arbejdsplanen for at akkomodere reparationer af kommunikationssystemet. Efter en del arbejde med problemet, døde
microcontrolleren der stod for kommunikationen. Vi var derfor nødt til at bruge ekstra tid på at udvikle en løsning der benyttede et andet udviklingskort.

Det var også meningen at vi skulle have forbedret afskærming af propellen der bliver brugt i opstilling, men efter mange forsøg var vi nødt til at opgive da 
3D printerne i DTU Fablab er ude af funktion.

\section*{Arbejdsprocessen}
Vi har fulgt den originale tidsplan så godt som muligt og endte ikke med at vige langt fra den. Der var selvfølgelig mindre afvigelser da microcontrolleren brød
sammen, men det blev løst med en ekstra arbejdsindsats før 3-ugers perioden. 

Vi har ført logbog hvor det var nødvendigt og brugt den som TODO liste.

\section*{Samarbejdet}
Nils har været til stor hjælp og har kunnet foreslå gode løsninger.

Samarbejdet i gruppen har fungeret godt. Dog bedst i starten hvor der hele tiden var arbejde nok til to mand. Da kommunikationen brød sammen var det Søren der
ordnede det da han har stor erfaring med den mikrocontroller der bliver brugt. Da kommunikationen er kritisk for at opstillingen virker, var der ikke meget andet
at lave i denne periode. I 3-ugers perioden gik arbejdsfordelingen dog tilbage til det originale niveau.

Mine primære arbejdsopgaver har været
\begin{itemize}
	\item Refaktorering og oprydningen af projektets kodebase ved projektets start.
	\item Implementering af kommunikation på linux siden.
	\item Implementering af Kalman filteret.
	\item Paper skrivning.
\end{itemize}

\section*{Evaluering}
Projektet er forløbet godt. Selvom vi startede med en opstilling der ikke fungerede og senere oplevere en kritisk komponentdød, formåede vi at
håndtere det uden at afvige betydeligt fra vores originale tidsplan. Det gik mindre godt at forsøge at 3D printe afskærmning til propellerne
da ingen af DTU Fablabs printere fungerede.

Vi er endt med at have produceret et altitude lokaliseringssystem der kan bruges i fremtidige projekter både af os selv og DTU automation. Hvilket
må siges at være en success der opfylder de ønsker vi formulerede problemstillingen ud fra.

\end{document}
