\documentclass[10pt,conference,draftcls]{IEEEtran}
\usepackage{fancyhdr}
%\usepackage[top=3cm, bottom=2cm, left=2cm, right=4cm]{geometry}
\usepackage[english]{babel}
\usepackage[utf8]{inputenc}
\usepackage[noadjust]{cite} % Bibtex package
\usepackage[cmex10]{amsmath}
\usepackage{amssymb}
\usepackage{amsfonts}
\usepackage{graphicx}
\usepackage{float}
%\usepackage{caption}
\usepackage[caption=false,labelfont=sf,textfont=sf]{subfig}
%\usepackage{lmodern}
\graphicspath{{./pictures/}}
%\usepackage{textpos}
\usepackage{color}
\usepackage{xcolor}
\usepackage{listingsutf8}
\usepackage{courier}
\usepackage{epstopdf}
\usepackage{fixltx2e}
\usepackage{pdfpages}
\usepackage{lipsum}
\usepackage{tikz}
\usetikzlibrary{external}\tikzexternalize
\usepackage{pgfplots}
\usepackage{svg}

%\epstopdfsetup{update,prepend,verbose,suffix=-generated} % use suffix because you don't want to accidentally overwrite a file that might have been a pdf source. The epstopdf package manual has more on that.

 
 \usepackage{hyperref}

%palatino
% Palatino for rm and math | Helvetica for ss | Courier for tt
%\usepackage{mathpazo} % math & rm
%\usepackage{mathrsfs}
%\linespread{1.05}        % Palatino needs more leading (space between lines)
%\usepackage[scaled]{helvet} % ss
%\usepackage{courier} % tt
%\normalfont
%\usepackage[T1]{fontenc}
\usepackage{bm}
\definecolor{javared}{rgb}{0.6,0,0} % for strings
\definecolor{javagreen}{rgb}{0.25,0.5,0.35} % comments
\definecolor{javapurple}{rgb}{0.5,0,0.35} % keywords
\definecolor{javadocblue}{rgb}{0.25,0.35,0.75} % javadoc
\definecolor{mauve}{rgb}{0.6,0,0}
%\definecolor{box}{rgb}{0.98,0.98,0.98} % box

%TESTTESTTEST

%setup listings
\lstset{language=C,
  extendedchars=true,
  %language=Octave,                % the language of the code
  basicstyle=\ttfamily\footnotesize,% the size of the fonts that are used for the code
  numbers=left,                   % where to put the line-numbers
  numberstyle=\tiny\color{gray},  % the style that is used for the line-numbers
  stepnumber=2,                   % the step between two line-numbers. If it's 1, each line 
                                  % will be numbered
  numbersep=5pt,                  % how far the line-numbers are from the code
  backgroundcolor=\color{white},      % choose the background color. You must add \usepackage{color}
  showspaces=false,               % show spaces adding particular underscores
  showstringspaces=false,         % underline spaces within strings
  showtabs=false,                 % show tabs within strings adding particular underscores
  frame=single,                   % adds a frame around the code
  rulecolor=\color{black},        % if not set, the frame-color may be changed on line-breaks within not-black text (e.g. comments (green here))
  tabsize=4,                      % sets default tabsize to 2 spaces
  captionpos=b,                   % sets the caption-position to bottom
  breaklines=true,                % sets automatic line breaking
  breakatwhitespace=false,        % sets if automatic breaks should only happen at whitespace
  title=\lstname,                   % show the filename of files included with \lstinputlisting;
                                  % also try caption instead of title
  keywordstyle=\color{blue},          % keyword style
  commentstyle=\color{javagreen},       % comment style
  stringstyle=\color{javared},         % string literal style
  escapeinside={\%*}{*)},            % if you want to add LaTeX within your code
  morekeywords={*,...},              % if you want to add more keywords to the set
  deletekeywords={...}              % if you want to delete keywords from the given language
}
\lstset{literate=%
{æ}{{\ae}}1
{å}{{\aa}}1
{ø}{{\o}}1
{Æ}{{\AE}}1
{Å}{{\AA}}1
{Ø}{{\O}}1
{°}{{${}^o$}}1
{I̅}{{$\overline{\mbox{I}}$}}1
{X̅}{{$\overline{\mbox{X}}$}}1
{V̅}{{$\overline{\mbox{V}}$}}1
{L̅}{{$\overline{\mbox{L}}$}}1
{D̅}{{$\overline{\mbox{D}}$}}1
{C̅}{{$\overline{\mbox{C}}$}}1
{M̅}{{$\overline{\mbox{M}}$}}1
{M̅}{{$\overline{\mbox{M}}$}}1
}

 \lstloadlanguages{% Check Dokumentation for further languages ...
         %[Visual]Basic
         %Pascal
         %C
         %C++
         %XML
         %HTML
         %Java
         VHDL
 }
%\DeclareCaptionFont{blue}{\color{blue}} 

%\captionsetup[lstlisting]{singlelinecheck=false, labelfont={blue}, textfont={blue}}
%\usepackage{caption}
\DeclareCaptionFont{white}{\color{white}}
\DeclareCaptionFormat{listing}{\colorbox[cmyk]{0.43, 0.35,
0.35,0.01}{\parbox{0.98\textwidth}{\hspace{15pt}#1#2#3}}}
\captionsetup[lstlisting]{format=listing,labelfont=white,textfont=white, singlelinecheck=false, margin=0pt, font={bf,footnotesize}}
%\lstset{language=Java}

\addto\captionsdanish{%
  \renewcommand{\abstractname}%
    {Abstract}%
}

%\setcounter{chapter}{1}
\setcounter{section}{0}

%\pagestyle{fancy}

%\lhead{31400 Electromagnetics assignment 3}
%\chead{ 22/11-13 }
%\chead[<ch-even>]{<ch-odd>}	
%\rhead{Carsten Nielsen s123161} %, \\ Søren Krogh Andersen s123369}

%\cfoot[\thepage]{\thepage}

%Macros
%By s�ren:
\newcommand\stdfig[4]{ %width,img,cap,label
\begin{figure}[h!]
\centering
\includegraphics[width=#1\textwidth]{#2}
\caption{#3} \label{#4}
\end{figure}
}
\newcommand\diff{\dot}
\newcommand\ddiff{\ddot}
\newcommand\facit[1]{\underline{\underline{#1}}}
\DeclareUnicodeCharacter{00A0}{ }

\begin{document}
\section*{Introduction}

Autonomous or semi-autonomous multi-rotor vehicles are subject to the 
same localisation problem that apply to all autonomous systems, with the added
difficulty that it must be performed in 3 dimensions instead of only 
the 2 that ground based vehicles can move in. 

Localisation can be performed by using sensor measurements to infer
ones position. Sensors that work well for flying vehicles are accelerometers,
rate gyroscopes and magnetometers. These sensors exist in micro electromechanical
(MEMs) versions that are small and light enough to fit on smaller aircraft.
Of these, only the magnetometer can provide
an absolute measure of position, but it is unuseable indoors or in any place
that has significant magnetic disturbances. A rate gyroscope measures the
rate of change of rotation around its axis and an accelerometer measures
acceleration, as its name implies. To obtain an absolute measure of position
from these devices, one must therefore integrate their outputs. This leads
to an unbounded error because the sensors are noisy. Other absolute measures 
of position, such as GPS, exist and have bounded errors, but very slow update
rates and thus do not lend themselves well to autonomous control. Luckily,
the aforementioned rate gyro and accelerometer have very fast update speeds.
Fusion of the fast sensors with unbounded error and the slow sensors with bounded
error can lead to a position estimate with fast update times and bounded error.
The best of both worlds.

A multi-rotor system can be simplified such that control only affects one dimension.
By attaching a rotor to a lever arm, the position of the rotor is practically limited
to a single dimension, when the lever arm is constrained to move within small angles.
This eliminates the need to use a gyroscope to compensate for the effect of gravity
on the accelerometer. This limits scope of localisation to dealing with noise
from the accelerometer and greatly simplifies the control of the system.

Our system is based upon the above description and consists of a single rotor attached
to a lever arm constrained to tilt within 0-30 degrees. The arm contains an accelerometer
and IR distance sensor near the rotor, and a potentiometer at the base that can measure
the angle of the arm.

\section*{Problem statement}
This project aims to answer the following question
\begin{itemize}
	\item How can a MEMs accelerometer be stabilized using mechanical noise rejection methods?
	\item How can measurements from a MEMs accelerometer, an IR distance sensor 
		and a potentiometer be combined to deliver fast and accurate position updates
		using a software filter?
	\item How can the position of the system be controlled using a PI, PD, PID controller?
		of the system.
	\item What software model accurately describes the physical system?
\end{itemize}
\end{document}
