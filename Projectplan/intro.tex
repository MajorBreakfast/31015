\section*{Introduction}

Autonomous or semi-autonomous multi-rotor vehicles are subject to the 
same localisation problem that apply to all autonomous systems, with the added
difficulty that it must be performed in 3 dimensions instead of only 
the 2 that ground based vehicles can move in. 

Localisation can be performed by using sensor measurements to infer
ones position. Sensors that work well for flying vehicles are accelerometers,
rate gyroscopes and magnetometers. These sensors exist in micro electromechanical
(MEMs) versions that are small and light enough to fit on smaller aircraft.
Of these, only the magnetometer can provide
an absolute measure of position, but it is unuseable indoors or in any place
that has significant magnetic disturbances. A rate gyroscope measures the
rate of change of rotation around its axis and an accelerometer measures
acceleration, as its name implies. To obtain an absolute measure of position
from these devices, one must therefore integrate their outputs. This leads
to an unbounded error because the sensors are noisy. Other absolute measures 
of position, such as GPS, exist and have bounded errors, but very slow update
rates and thus do not lend themselves well to autonomous control. Luckily,
the aforementioned rate gyro and accelerometer have very fast update speeds.
Fusion of the fast sensors with unbounded error and the slow sensors with bounded
error can lead to a position estimate with fast update times and bounded error.
The best of both worlds.

A multi-rotor system can be simplified such that control only affects one dimension.
By attaching a rotor to a lever arm, the position of the rotor is practically limited
to a single dimension, when the lever arm is constrained to move within small angles.
This eliminates the need to use a gyroscope to compensate for the effect of gravity
on the accelerometer. This limits scope of localisation to dealing with noise
from the accelerometer and greatly simplifies the control of the system.

Our system is based on previous work done by \emph{Martin Rygaard Hermansen} and documented in his Bachelor's Thesis: \emph{Position Control of Simplified, Tethered Drone}. His system consists of a single rotor attached
to a lever arm constrained to tilt within 0-30 degrees. The arm contains an accelerometer
and infra-red distance sensor near the rotor, and a potentiometer at the base that can measure the angle of the arm. 
However he had problems when utilizing the accelerometer, presumably due to mechanical noise, which is noise caused by the construction to which the accelerometer is attached.

We seek to solve the problems associated with the use of an accelerometer in this setup and dessign an altitude controller for the simpified system using sensor fusion of accelerometer and one or more other sensors.
