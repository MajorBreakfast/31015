\section{Discussion}\label{sec:discussion}
We determined that the ultrasound proximity sensor was the best choice for near earth altitude estimation for a real multirotor 
vehicle. If greater accuracy and less oscillation around the set point is required, a different approach to handling the noise would
need to be developed. However, we feel that an accurary of \(\pm\)1cm is more
than adequate.

The state estimation that we have implemented is intended to be used on a multirotor vehicle. In it's current state it can be applied
to estimate the altitude of such a platform without any modification. Only the control of the motors would need to be different, as 
a full multirotor system 
of course has different flight dynamics than a tethered lever arm. For longitudinal and latitudinal position estimation, the same method,
using proximity sensors, could be applied. It is however unlikely that this would yield good results because objects in this plane usually
have gaps between them that would confuse the system. Instead we think that visual odometry\cite{Nister2004} is a much more
appealing technology for this
purpose. It could also be interesting to compare distance estimates obtained through stereoscopic vision with those of the proximity sensor.

We envision a multirotor with a MEMs accelerometer and rate gyroscope
mounted at the center of rotation together with an ultrasound proximity sensor, GPS module and camera for visual odometry.
Using our state estimation implementation,
it would be possible to rely on the proximity sensor and visual odometry system, when operation near a surface plane, and rely
on the GPS which has much larger 
measurement variance, when operating out of proximity range.

Such a system would be able to perform fine grained altitude maneuvers when near objects where this ability is critical, yet also capable
of traversing larger open spaces between objects where such precision is not necessary.
