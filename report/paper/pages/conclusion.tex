\section{Conclusion}\label{sec:conclusion}
Autonomous or semi-autonomous airborne robots must be able to locate and control their position in three dimensions. We have designed
and implemented an alititude estimation scheme for near surface flights using either IR or ultrasound proximity sensors together with 
a MEMs accelerometer and rate gyroscope. We have used the implementation in order to control a simplified multirotor model, a lever arm with a propeller 
on one end, inherited from a previous bachelor thesis by Hermansen. Our implementation solves the problem of providing high frequency altitude estimation
allowing for high bandwidth control. Our implementation is accurate to within \(\pm\)1cm and will function on any platform
operating near a surface, provided it contains an accelerometer, rate gyroscope and proximity sensor.

We observe that the ultrasound 
sensor is better suited for outdoor applications and provides better accuracy for the models we have tested.

In order to control a full multirotor vehicle in all dimensions, the system must be expanded such that it can handle localisation 
along the longitudinal and latitudinal axes in addition to the development of a controller for the non-simple case of only a single tethered rotor. 
We propose investigating the possibility of integrating visual odometry into the system to perform localisation.
