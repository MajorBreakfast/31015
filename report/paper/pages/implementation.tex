\section{Implementation}
Our implementation consists of a realtime linux desktop machine, handling the localisation and control logic, and a microcontroller
positioned on the arm. The microcontroller handles measurements from the MEMs accelerometer and rate gyroscope as well as the ultrasound
distance sensor. The IR distance sensor is connected directly to the desktop machine through a National Instruments ADC/DAC, as is the 
potentiometer at the base of the arm which is used for reference measurements.

We have made several changes to the original setup by Hermansen. The most important of these are

\begin{itemize}
		\item A complete overhaul of the communications subsystem in order to support sending accelerometer, rate gyroscope and 
			distance measurements at 1kHz and avoid intermittent transmission errors. 
		\item The original complementary filter has been replaced by a static kalman filter for state estimation and a linear 
			quadratic regulator (LQR)\cite{Hendricks2008} for motor command control.
			\item An ultrasound sensor has been added to the arm for distance measurements.
\end{itemize}

The microcontroller on the arm is an ARM® Cortex M4F mounted on a Tiva Launchpad development board.
It is connected to an MPU9150 on which the MEMs accelerometer and rate gyroscope resides.

\subsection{Communications subsystem}
The Tiva Launcpad recieves measurements from the accelerometer, rate gyroscope and ultrasound sensor. The accelerometer and 
rate gyroscope is sampled at 1kHz, while the ultrasound sensor has a varying sample rate that depends on the distance to the nearest
object. The measurement data is transformed into a packet where the leading bit of every byte contains a synchronization bit. When
the bit is set it signifies the start of a packet. These packets are then sent to the desktop machine at 1kHz. When the ultrasound 
sensor is not ready with a new measurement, zeroes are transmitted. The packets are received on the desktop machine, which is 
running a control loop at 200Hz. Here the packets are unpacked to their original form and corrupted packets are thrown away.
\subsection{Localisation}
Our localisation algorithm is a three step process. First the rate gyroscpoe measurements are used to determine the orientation of
the arm. This is done using the Madgwick AHRS system\cite{Madgwick2011}. It is necessary to know the orientation of the arm before
the acceleration can be calculated because the acceleration measured by the tri-axis MEMs accelerometer is relative to the 
accelerometer frame. Additionally, gravity results in an offset in the accelerometer measurements that has constant magnitude, but
orientation varying with the orientation of the accelerometer. The acceleration is then rotated into earth frame and the vertical
acceleration is isolated since we only care about the height of the arm.
This is then used as the control input to a static multi rate kalman filter\cite{Welch2006}. Because the filter is static we
only predict the new state 
\begin{equation*}
	\hat{\mathbf{x}}_{k\vert k-1} = \mathbf{F}\hat{\mathbf{x}}_{k-1\vert k-1} + \mathbf{B}a
\end{equation*}
Compute the measurement error 
\begin{equation*}
	y = z_k - \mathbf{H}\hat{\mathbf{x}}_{k\vert k-1} 
\end{equation*}
And update the state
\begin{equation*}
	\hat{\mathbf{x}}_{k\vert k} = \hat{\mathbf{x}}_{k\vert k-1} + \mathbf{K} y
\end{equation*}

The kalman filter executes a predict step for every measurement coming from the accelerometer at 1kHz and an update step as soon 
as a measurement is received from the distance sensor, be it ultrasound or infrared, at \(\sim\)30Hz. When using the IR distance 
sensor a measurement is always ready, however it is only updated at the \(\sim\)30Hz. With the ultrasound sensor however, a measurement
is only valid in the same sample it is recorded. The ultrasound sensor is also afflicted by intermittent noise that results false readings,
typically far away from previous values. We employ a moving average filter that culls out measurements too far from the current average.
Let \(x_k\) be the current measurement sample, \(w\) the number of measurements taken into the moving average and \(Z\) the threshold value.
A measurement is accepted if 
\begin{equation*}
	Z >\frac{1}{w} \lvert \sum_{i=k-w}^{k}x_i \rvert
\end{equation*}
This improves stability greatly as we rely somewhat heavily on the low frequency distance measurements in the kalman filter.

For the kalman filter we use the state space vector where the acceleration, velocity and position are perpendicular to the ground.
\begin{equation*}
	\hat{\mathbf{x}} = \begin{bmatrix}
		a_{off} \\
		v \\
		x
	\end{bmatrix}
\end{equation*}
We include the term \(a_{off}\) because the accelerometer is not only affected by random noise with zero mean, but also a 
steady bias that depends on temperature as well as other factors. The \(a_{off}\) term is an offset that is modified along
with the velocity and position to yield a more accurate state estimation.
The state transition model where \(T\) is the sample period for the accelerometer.
\begin{equation*}
	\mathbf{F} = \begin{bmatrix}
		1 & 0 & 0 \\
		T & 1 & 0 \\
		0 & T & 1 
	\end{bmatrix}
\end{equation*}
Our control input model, where the control input is actually the acceleration is.
\begin{equation*}
	\mathbf{B} = \begin{bmatrix}
		0 \\
		T \\
		0 
	\end{bmatrix}
\end{equation*}
And since our measurements give us the absolute height of the arm the observation matrix becomes:
\begin{equation*}
	\mathbf{H} = \begin{bmatrix}
		0 \\
		0 \\
		1
	\end{bmatrix}
\end{equation*}
Because the kalman filter is static, we need different kalman gains when using the potentiometer (pot), ultrasound sensor (ult) and IR sensor (IR).
We have determined the following kalman gains by manual tuning.
\begin{equation*}
	\mathbf{K}_{pot} = \begin{bmatrix}
		0.025 \\
		0.125 \\
		0.125 
	\end{bmatrix},\quad
	\mathbf{K}_{ult} = \begin{bmatrix}
		0.01 \\
		0.05 \\
		0.05 
	\end{bmatrix}, \quad
	\mathbf{K}_{IR} = \begin{bmatrix}
		0.01 \\
		0.05 \\
		0.05 
	\end{bmatrix}
\end{equation*}

The output from the kalman filter is used to contruct a full one dimensional state estimate in the form
\begin{equation*}
	\hat{\mathbf{x}} = \begin{bmatrix}
		x \\
		v \\
		a
	\end{bmatrix}
\end{equation*}
Where \(x\) is the estimated height, \(v\) the velocity along the vertical axis and \(a\) the acceleration along the vertical axis. 
This state estimate is then used with an LQR controller to compute the control output for the motors.
\subsection{Controller}
We employ a static LQR controller with the following gain constants.
\begin{equation*}
	\mathbf{K}_{LQR} = \begin{bmatrix}
		1000 \\
		275 \\
		0 
	\end{bmatrix}
\end{equation*}
The controller is essentially a PD controller since it uses only the current position and the derivative thereof, velocity, to calculate
the next control signal.
