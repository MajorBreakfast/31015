\section{Introduction}
Autonomous or semi-autonomous airborne robots must be able to locate and control their position in three dimensions instead of
the two required for ground-based robots. Various sensors can be used to provide a position estimate and thereby perform localisation.
There are only a few sensors that work well for lightweight aerial vehicles, e.g micro electromechanical (MEMs) accelerometers, 
gyroscopes and magnetometers, IR distance sensors, ultrasound distance sensors, gps and small cameras. Of these options, only the MEMs
sensors can deliver very fast measurements at a rate larger than 200Hz except in rare cases. It would therefore seem that the MEMs
sensors are the easy choice for any application, but since the accelerometers have noise and acceleration measurements need to be 
integrated twice to yield a position estimate, they are plagued by an unbouded error. Furthermore the available MEMs gyroscpess are rate 
gyroscopes, that is to say they measure the rate of rotation instead of absolute orientation. Again these have noise and their measurements
must be integrated to yield an estimate of orientation, leading to unbounded error. This leaves the MEMs magnetometer as a source of
absolute orientation measurements, but using the magnetometer in this way requires precise knowledge of how the magnetic field is 
shaped whereever the robot goes. This is not possible indoors or near large structures.
It is therefore necessary to use some of the sensors with slower sample rates, in conjuction with
the MEMs accelerometer and rate gyroscope. The unbounded error in the fast position estimate, derived from the accelerometer and gyroscope,
can then be corrected at the sample rate of the slower sensor. This allows a position estimate with bounded error that is also updated
frequently. 

We aim to develop and implement such a position-control loop for use on a multirotor drone. However, the development of the position 
estimation and proof of concept control does not require the full multirotor system. We have therefore implemented the position estimation 
and control, using a MEMs accelerometer and gyroscope together with a stiance sensor (ultrasound or IR), for a simplified multirotor model: 
A lever arm with a single rotor attached to one end. The arm also contains a MEMs accelerometer and rate gyroscope near the rotor.

The setup is inherited from Hermansen who built it as part of his bachelor thesis\cite{Hermansen2013a}.
Like us, Hermansen had intended to control the arm using the accelerometer. However, he determined that there was too much mechanical 
noise affecting the system, making the accelerometer unuseable, and likely ran out of time before being able to dig deeper into the 
issue.

Upon beginning this project we intended to start by solving the mechanical noise problem. It quickly became apparent that the 
accelerometer issue was not caused by noise, but by communication error between the accelerometer and desktop machine used in the setup.
Our focus therefore shifted towards creating a reliable communications link instead of eliminating mechanical noise.

In this paper we present a localisation system that uses a MEMs accelerometer and rate gyroscope as well as any distance measuring sensor,
that together with a linear quadratic gaussian controller is able to control the lever arm. The localisation implementation is described
in section~\ref{sec:localisation} and the controller in section~\ref{sec:controller}.
We have also implemented a new communications 
link between the MEMs accelerometer, the rate gyroscope and the mission control unit. This was done to eliminate the data corruption 
issues faced by Hermansen as well as to be able to provide enough bandwidth to transmit accelerometer, rate gyroscope and distance 
measurements at 1kHz. The details are briefly explained in section~\ref{sec:comms}. 


%TODO:
%INSERT FINAL SPECS HERE
%INSERT STRUCTURE OF PAPER
